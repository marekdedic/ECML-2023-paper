\section{Related work}

The publications most relevant to our research are \cite{chen_harp_2018}, in which the HARP approach is proposed, and \cite{gasteiger_diffusion_2019}, in which graph coarsening is performed by means of graph diffusion. Because we directly extend, modify or combine those methods, they will be recalled in some detail in Sections \ref{sec:harp} and \ref{sec:gdc-coarsening}. Other important works concerning graph coarsening are \cite{akyildiz_understanding_2020,chen_graph_2022,cai_graph_2022}, which survey numerous coarsening methods, \cite{huang_scaling_2021}, which presents results concerning scalability of graph coarsening, \cite{catalyurek_multithreaded_2012,herrmann_multilevel_2019}, which establish coarsening as a basis for partitioning, and \cite{loukas_graph_2019}, which shows relationships of graph coarsening to properties of the Laplacian. In view of the fact that the HARP approach, which we extend and modify, is a multilevel approach, we paid attention also to the multilevel graph coarsening methods proposed in \cite{bethune_hierarchical_2020,xie_graph_2020,zhang_harp_2021,liu_hierarchical_2021}, among them \cite{zhang_harp_2021} also being inspired by HARP.

In a broader context, our research is related to the more general topic of graph reduction, which apart from graph coarsening includes also graph sparsification. A general framework covering both coarsening and sparsification has been proposed in \cite{bravo_hermsdorff_unifying_2019}. Also of note is the recent work \cite{kammer_space-efficient_2022}, presenting an alternative coarsening approach for planar graphs and \cite{liu_comprehensive_2022}, which sparsifies not only the graph topology, but simultaneously also the features of its nodes and weights of graph neural network used for its embedding. Elaboration of graph coarsening methods in machine learning can built on several decades of their succesful application, such as pairwise aggregation, independent sets, or algebraic distance, in numerical linear algebra \cite{chen_graph_2022}, including in particular multilevel graph coarsening \cite{osei-kuffuor_matrix_2015,ubaru_sampling_2019}.
