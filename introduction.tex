\section{Introduction}
Across a wide variety of applications and domains, graphs emerge as a domain-independent and ubiquitous way of organizing data. Consequently, machine learning on graphs has, in recent years, seen an explosion in popularity, breadth and depth of both research and applications. While there have been significant advances in algorithms for learning from graph data \cite{kipf_semi-supervised_2016, defferrard_convolutional_2016, li_deepergcn_2021}, the structure of the underlying data has, until recent works \cite{gasteiger_diffusion_2019, topping_understanding_2021, velickovic_geometric_2021, chamberlain_grand_2021}, received much less attention. In this work, we aim to take a closer look at the importance of the individual nodes and neighbourhoods that form a graph from the point of view of downstream tasks.

Typically, an application of machine learning to graphs has two phases: representation learning, which maps the graph into a Euclidean space, and a downstream task, such as classification, regression, or clustering. The first phase has very high computational demands, which can be substantially decreased with graph coarsening. However, it is known that there is an interplay between coarsening and the quality of embedding \cite{akyildiz_understanding_2020, makarov_survey_2021}, which in turn entails an interplay between coarsening and the quality of the downstream task.

Our work builds on the HARP \cite{chen_harp_2018} method for pretraining on coarsened graphs. In HARP, a graph is repeatedly coarsened and the coarser graphs are then used in reverse order (from coarsest to finest) to pre-train a graph representation learning algorithm. While HARP itself works with and modifies the graph structure, this is not the main interest of its authors, who focus more on the obtained representation of the original graph. In our work, we aim to leverage such a graph coarsening approach to study the properties of graphs and graph embedding models from the point of view of the graph structures getting coarsened with the additional benefit of decreasing computational requirements. We modify and generalize the HARP framework to closely study the relationship between graph coarsenings and graph quality in terms of the performance of a downstream task. In our case, we chose transductive node classification as the ultimate task, however, the presented algorithms are general graph representation learners that can be utilized for a wide variety of tasks.

\todo[inline]{Relation to ITAT paper?}

The main contributions of this work are the general framework for graph coarsening and extensions of the HARP algorithm. We extend both main parts of the algorithm in order to observe the effect of graph coarsening to model quality on a fine level. We present two alternative graph coarsening schemes based on graph diffusion convolution and evolutionary algorithms. Additionally, we present a novel way for un-coarsening the reduced graph in a targeted way based on the confidence of downstream classification for particular nodes.

In the next section, related work and its relationship to this work are discussed. Following that, in Section \ref{sec:harp-framework}, HARP, the method our work builds on, is presented, together with our extension of it into a general framework for graph coarsening. Section \ref{sec:our-method} is the core of this work, presenting first our extension of the prolongation step, as well as our proposed alternative graph coarsening schemes. Finally, these proposals are experimentally verified and compared in Section \ref{sec:experimental-evaluation}.
