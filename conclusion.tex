\section{Conclusion}

In this work, a general graph coarsening schema modelled on the basis of the HARP algorithm was proposed. A novel approach to prolonging graphs in the HARP setting was presented, yielding an adaptive algorithm that selectively prolongs the graph in a way that maximizes performance of the considered downstream task under limited graph size. Additionally, 2 alternatives to HARP coarsening based on graph diffusion were presented. Together, these two improvements substantially increase the versatility of HARP, turning it from a method for pre-training into a framework for graph reduction. Such a framework enables the study of properties of particular graphs, making it possible to reveal global structures. The framework may be used for lowering computational demands while preserving downstream task performance as well.

All of the proposed methods were experimentally verified. While the behaviour differs between the graphs studied, in general, our experiments reveal that at about 50\% reduction in node count, the accuracy was still reasonably close to the accuracy on a full graph for most datasets. Additionally, the coarsenings based on graph diffusion were shown to outperform the original coarsening, again with the exact difference depending on the particular dataset and coarsening level.

In future work, a simpler, direct way of tackling the performance-complexity trade-off problem for graphs may be studied as an alternative to the approach proposed in this work. Our preliminary exploration of this direction \cite{prochazka_downstream_2022} studies the setting of direct adaptive coarsening, instead of a fixed coarsening followed by adaptive prolongation.
