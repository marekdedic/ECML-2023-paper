\section{HARP extension for flexible performance-complexity balancing}

Graph-based methods such as node2vec typically have a large number of parameters -- on the widely used OGBN-ArXiv dataset (see \cite{hu_open_2021}), the state-of-the-art node2vec model has over 21 million parameters. At the same time, recent works in the domain of graph learning have started to focus more heavily on simpler methods as a competitive alternative to heavy-weight ones (see \cite{frasca_sign_2020,huang_combining_2020,salha_keep_2019,zhang_eigen-gnn_2021}). As the authors of \cite{chen_harp_2018} observed, HARP improves the performance of models when fewer labelled data are available. The proposed lower complexity models based on HARP could also improve performance in a setting where only low fidelity data are available for large parts of the graph. Coarser models could be trained on them, with a subsequent training of finer models using only a limited sample of high fidelity data.

In this work, we extend the general HARP framework to study the preformance-complexity characteristics of graph data. To this end, we propose alternatives to both the coarsening as well as the prolongation step of HARP. First, in Section \ref{sec:adaptive-prolongation}, we replace the simple prolongation approach by an adaptive prolongation algorithm. Second, in Section \ref{sec:coarsening-algorithms}, we study two alternative ways of coarsening the graph.

\subsection{The adaptive prolongation approach}\label{sec:adaptive-prolongation}

In standard HARP, once the coarsened graphs are obtained, the way to train the graph embedding is fairly straightforward. Starting with the coarsest graph, an embedding model such as node2vec is trained for a set amount of training time (i.e. epochs). Following that, a step to a graph that is one level finer is made. The embedding learned on the immediately preceding coarser graph is \name{prolonged} to the embedding of the following finer graph, in which the representations of merged nodes are copied and reused. Then, with this prolonged embedding as the starting state, the embedding algorithm continues training and this process is repeated until reaching the original graph.

While this style of prolongation is fine when HARP is used only as a means of pre-training, this approach is far too crude when studying the relationship between graph complexity and the quality of graph embedding and subsequent downstream applications. For example, the widely-used Cora dataset \cite{yang_revisiting_2016} has in its original form 2708 nodes, while the graph resulting from one application of the HARP coarsening schema has only about 1100 nodes (exact numbers may differ run-by-run). Such a relatively high reduction ratio effectively prevents any sufficient understanding of the relationship between graph reduction and changes in the quality of its embedding.

In order to offer a more fine-grained observation of the graph complexity and its effect on the downstream task, we present the adaptive prolongation approach. This algorithm works with the pre-coarsened graphs produced for example by HARP, however, the embedding is learned in a different manner.

\begin{algorithm}
  \caption{Adaptive prolongation}
  \label{alg:adaptive-prolongation}
  \begin{algorithmic}
    \Require $ G $ \Comment The original graph
    \Require $ \mathvec{e}_{i+1} $ \Comment The previous embedding
    \Require $ \mathvec{y}_\mathrm{train} $ \Comment Training labels
    \Require $ replacement\_maps $ \Comment A list of records of all the original graph coarsenings
    \Require $ n_p $ \Comment The number of nodes to prolong
    \Ensure $ merges\_to\_prolong $ \Comment A list of node merges to be prolonged (\enquote{undone})
    \Ensure $ new\_replacement\_maps $ \Comment Updated coarsening records without the prolonged merges
    \Statex
    \State $ node\_order = \Call{get\_node\_order}{G, \bm{e}_{i+1}, \bm{y}_\mathrm{train}, replacement\_maps} $
    \State $ merges\_to\_prolong \gets \Call{select\_nodes}{node\_order, n_p, replacement\_maps} $
    \State $ new\_replacement\_maps \gets \Call{undo\_merges}{replacement\_maps, merges\_to\_prolong} $
    \Statex
    \Function{get\_node\_order}{$ G, \bm{e}_{i+1}, \bm{y}_\mathrm{train}, replacement\_maps $}
        \State $ \bm{e}_0^\mathrm{temp} \gets \Call{fully\_prolong\_embedding}{\bm{e}_{i+1}, replacement\_maps} $
        \State $ model \gets \Call{train\_downstream\_model}{\bm{e}_0^\mathrm{temp}, \bm{y}_\mathrm{train}} $
        \State $ entropy\_per\_node \gets H \left( \Call{predict}{model, node} \right) $ for each $ node \in V \left( G \right) $
        \State \Return $ V \left( G \right) $, sorted in descending order by $ entropy\_per\_node $
    \EndFunction
    \Statex
    \Function{select\_nodes}{$ ordered\_nodes, n_p, replacement\_maps $}
        \State $ selected\_merges \gets \left\{ \right\} $
        \For{$ node \in ordered\_nodes $, until $ \left\lvert selected\_merges \right\rvert = n_p $}
            \State $ merge \gets \Call{resolve\_merge}{node, selected\_merges, replacement\_maps} $
            \State If $ merge \neq \mathrm{null} $, add $ merge $ to $ selected\_merges $
        \EndFor
        \State \Return $ selected\_merges $
    \EndFunction
    \Statex
    \Function{resolve\_merge}{$ node, already\_selected\_merges, replacement\_maps $}
        \State $ merge \gets \mathrm{null} $
        \For{$ replacement\_map \in replacement\_maps $ from finest graph to coarsest}
            \State $ merge\_candidate \gets $ find in $ replacement\_map $ a merge that affects $ node $, continue with next $ replacement\_map $ if not found
            \If{$ merge\_candidate \in already\_selected\_merges $}
                \State \Return $ merge $
            \EndIf
            \State $ merge \gets merge\_candidate $
            \State Apply $ merge $ to $ node $, so that in the next loop, a subsequent merge may be selected
        \EndFor
        \State \Return $ merge $
    \EndFunction
  \end{algorithmic}
\end{algorithm}

The adaptive prolongation approach (Algorithm \ref{alg:adaptive-prolongation}) uses the pre-computed coarsened graphs as a way to progressively increase the number of nodes with which the embedding is trained. However, its iterations of embedding training and prolongation are decoupled from the pre-computed coarsened graphs. Instead, in each step of the training, the current embedding is used to train a node classifier that guides which nodes should be prolonged. The node classifier is equivalent to the one which is to be eventually used for the downstream task and is trained on the same training subset of graph nodes. A measure guiding the prolongation is to be produced using this classifier -- ideally, only the nodes where the classifier performs the worst should be prolonged. To this end, the confidence of the classifier for each node is used. In our experiments, several methods of assessing the confidence were tried, ultimately settling on the entropy of the output of the softmax layer for each node -- representing the amount of information the classifier is able to infer about each node. Starting with nodes with the highest entropy, for each node the pre-computed coarsenings are searched for edge contractions involving the node, with preference for contractions from later steps of the repeated coarsening (corresponding to coarser graphs). A given number of such edge contractions is selected and undone in each prolongation step, gradually advancing from the coarsest graph to the original, finest one.\todo{Lukáš: Padají tu z nebe parametry, je potřeba ta rozhodnutí odůvodnit.}

\subsection{More general approaches to coarsening}\label{sec:coarsening-algorithms}

While the adaptive prolongation approach substantially generalizes the original method into a powerful tool for studying and leveraging graph structure and its properties under a coarsening, it still relies on the pre-computed coarsenings to guide the prolongation process. In this Section, we first present a brief overview of the coarsening algorithm as proposed by \cite{chen_harp_2018}, followed by two alternative proposals for ways in which the coarsening can be computed.

\subsubsection{HARP coarsening}\label{sec:harp-coarsening}

The authors of \cite{chen_harp_2018} introduce two particular coarsening methods that together realize the function \( \psi_i \) from Section \ref{sec:harp} -- \textbf{edge collapsing} and \textbf{star collapsing}. Edge collapsing is a very simple method -- out of all the edges \( E \left( G \right) \), a maximal subset \( E' \) is selected randomly such that no two edges from \( E' \) are incident on the same node. Then, each edge in \( E' \) is contracted.

The edge collapsing algorithm is a good general way of lowering the number of nodes in a graph, however, some structures are not easily collapsed by it. An example of such a structure is a \enquote{star} -- a single node connected to many other nodes. To coarsen graphs with such structures effectively, the star collapsing algorithm is proposed. For each such \textit{hub} node \( u \) in order of decreasing degree, its unconnected neighbouring nodes are taken and merged pairwise. All edges incident on such nodes are replaced with edges incident on the corresponding newly created nodes. As in edge collapsing, nodes to be merged are selected in such a way that no node is merged twice.

These two approaches are combined, with each HARP coarsening step being a star collapsing step followed by an edge collapsing step. Of a particular note is the fact such a coarsening scheme doesn't follow the definition presented in Section \ref{sec:coarsening-properties}. The star collapsing algorithm merges nodes that are adjacent to a common hub node, however, these nodes need not be connected by an edge. In our previous work\todo{Šlo by tady ocitovat loňský paper? Tím bychom tohle měli z krku... Akorát byl pouze na DDnech (dát na ArXiv?)}, we experimentally verified that the star collapsing algorithm can be replaced by a similar algorithm that merges nodes adjacent on a hub node with the hub node itself. Such a replacement modifies the HARP coarsening scheme to be in line with the definition presented in Section \ref{sec:coarsening-properties}.

\subsubsection{Graph diffusion coarsening}

Our definition of a graph coarsening requires choosing some edges from the original graph. Intuitively, one way of constructing a graph coarsening would be to merge nodes which are similar and therefore no significant amount of information is lost due to such a coarsening. Following both of these premises, a coarsening based on graph diffusion is proposed based on the Graph Diffusion Convolution (GDC) \cite{gasteiger_diffusion_2019} algorithm. In an overview, we define a generalized graph diffusion matrix
\begin{equation}\label{eq:gdc-matrix}
  \mathmat{S} = \sum_{k = 1}^\infty \theta_k \mathmat{T}^k
\end{equation}
such that the power series converges. The parameters \( \theta_k \) together with the generalized transition matrix \( \mathmat{T} \) define the exact way in which the diffusion is achieved. Among the choices for \( \mathmat{T} \) is the random walk transition matrix \( \mathmat{T}_ \mathrm{rw} = \mathmat{A} \mathmat{D}^{-1} \) and the symmetric transition matrix \( \mathmat{T}_\mathrm{sym} = \mathmat{D}^{-\frac{1}{2}} \mathmat{A} \mathmat{D}^{-\frac{1}{2}} \) where \( \mathmat{D} \) is the diagonal matrix of node degrees. Additionally, \( \mathmat{T}_\mathrm{rw} \) is taken to be column-stochastic.

Two special cases of this general schema are the Personalized PageRank algorithm (PPR) \cite{page_pagerank_1999} and the heat kernel \cite{kondor_diffusion_2002}. PPR correspond to choosing
\[ \mathmat{T} = \mathmat{T}_\mathrm{rw} \]
\[ \theta_k = \alpha \left( 1 - \alpha \right)^k \]
where the parameter \( \alpha \in \left( 0, 1 \right) \) is the teleport probability. The heat kernel corresponds to choosing
\[ \mathmat{T} = \mathmat{T}_\mathrm{rw} \]
\[ \theta_k = e^{-t} \frac{t^k}{k!} \]
with the parameter \( t \) being the diffusion time. For both of these special cases, Equation \ref{eq:gdc-matrix} has a closed-form solution. The result of most diffusion processes (including PPR and the heat kernel) is a dense matrix \( \mathmat{S} \), which then needs to be sparsified. In GDC, two methods of sparsification are considered -- thresholding the matrix values and selecting top-\( k \) entries for each column of the matrix. After sparsification, the transition matrix is normalized in a similar way to the input adjacency matrix, i.e. by rows, columns or symmetrically.

To apply this algorithm as a way of coarsening the graph, the edge set produced by GDC with sparsification is intersected with \( E \left( G \right) \) and the resulting edges contracted in the graph, therefore conforming to our graph coarsening definition. Selection of the parameters of the method is further discussed in Section \todo{ref}.

\subsubsection{Evolved coarsening}
\todo[inline]{Fill this in}
